\section{Análise Teórica}

Nesta secção, vão ser mostradas todas as etapas da análise teórica para o circuito Complementary Folded-Cascode (CFC)
operational transconductance amplifier (OTA). O objetivo com esta análise é determinar os valores dos diferentes parâmetros, 
as respetivas demonstrações para efeitos de otimização, conseguindo assim cumprir os objetivos que foram definidos na tabela \ref{tab:specs}.

Para analisar teóricamente o CFC OTA, usámos como recurso uma montagem que já nos era conhecida o convencional amplificador Folded Cascode, mostrado
na Figura \ref{fig:Cascode}, o CFC parte da mesma topologia, mas tem algumas modificações levando a grandes melhorias.

\textcolor{red}{Explicaçãozinha sobre as melhorias do CFC para o normal}


\begin{figure}[htp]
    \centering
    \includegraphics[width=10cm]{Images/Fcascode.png}
    \caption{Amplificador Folded Cascode}
    \label{fig:Cascode}
\end{figure}

Ao olharmos para o circuito e como foi estudado este é diferencial, portanto podemos apenas estudar metade do circuito,
devido à sua simetria, o circuito é possível ser visto na Figura \ref{fig:Circ_Simp}.

\begin{figure}[htp]
    \centering
    \includegraphics[width=15cm]{Images/Circ_Analisado_Teorica.png}
    \caption{CFC OTA simplificado}
    \label{fig:Circ_Simp}
\end{figure}

Na Figura \ref{fig:SimplificadoMaximo} está representado o circuito simplificado, este que foi escolhido devido a ter 
o nó de saída, ficando mais simples de retirar os parâmetros e de analisar o circuito. 
Nas subsecções abaixo constaram as explicações de como foram retiradas as expressões e os valores que foram meencionados ao longo 
do relatório.

\begin{figure}[htp]
    \centering
    \includegraphics[width=15cm]{Images/Simplificadao.png}
    \caption{Circuito Simplificado}
    \label{fig:Circ_Simp}
\end{figure}


\subsection{Ganho DC}

Tal como foi estudado o ganho DC $A_v$ é obtido pelo produto da transcondutância de curto circuito $G_M$ e
a resistência de saída $R_{out}$.

\begin{align}
    A_v = G_M \cdot r_{out} \,\text{[V/V]}
    \label{eq:ganho1}
\end{align}

Os valores de $G_M$ e $R_{out}$ podem ser obtidos pela análise 
do modelo em pequenos sinais do circuito simplificado da Figura \ref{fig:Circ_Simp}.
No entanto, para efeitos de simplificação, os valores para $G_M$ e $R_{out}$ foram 
obtidos através de aproximações, levando a expressões mais simples e sem a necessidade
de análise do circuito em pequenos sinais.

As seguintes expressões permitem determinar $G_M$ e $R_{out}$:
\begin{align}
    Gm = -(g_{m2a} + g_{m2b})   \, [S]
    \label{eq:gm}
\end{align}

\begin{align}
    R_{out} = r_{op} // r_{on} = \frac{1}{g_{op} + g_{on}} \, [\Omega]
    \label{eq:rout}
\end{align}

Onde $g_{op}$ e $g_{on}$ são obtidos através das expressões:

\begin{align}
    g_{on} = \dfrac{g_{gds4P}}{g_{m4P}} \times ({g_{gds2aN} + g_{gds11P}}) \, [S]
    \label{eq:gon}
\end{align}

\begin{align}
    g_{op} = \dfrac{g_{gds6N}}{g_{m6N}} \times ({g_{gds2bP} + g_{gds8N}})   \, [S]
    \label{eq:gop}
\end{align}

Da equação inicial do ganho DC, substituindo pelas expressões correspondentes de $G_M$ e $R_{out}$, obtemos a seguinte expressão final, 
já simplificada, para o ganho DC do circuito CFC OTA:

\vspace{\baselineskip}
\begin{align}
A_v=\frac{
    \big(g_{m4P} \cdot g_{m6N}\big) \big( BE \cdot g_{m2aN} + BE \cdot g_{m2bP} \big)
}{
    BE \cdot g_{m6N} \cdot g_{ds4P} \cdot \big( g_{ds11P} + g_{ds2aN} \big)
    + BE \cdot g_{m4P} \cdot g_{ds6N} \cdot \big( g_{ds2bP} + g_{ds8N} \big) 
} \, [V/V]
    \label{eq:ganhoFinal}
\end{align}

 A constante BE representa o efeito de corpo dos transístores em questão.

 

\subsection{Produto de ganho e largura de Banda (GBW)}
O produto de ganho e largura de banda (GBW) é obtido sabendo o ganho em DC e a frequência 
do polo dominante do circuito, ou seja, a frequência para quando o ganho do amplificador é de 0dB. 

A expressão correspondente ao polo dominante é dada por:
\begin{align}
    f_{p1} = \frac{1}{2 \pi \times R_{out} \times C_{out}}  \, [Hz]
    \label{eq:fp1}
\end{align}

Onde $C_{out}$ é a capacidade vista do nó de saída do circuito e $R_{out}$
a resistência vista nesse mesmo nó. 

Sabendo que a expressão do GBW é dada por:
\begin{align}
    GBW = \frac{A_v}{f_{p1}}    \, [Hz]
    \label{eq:GBW}
\end{align}

E sabendo as expressões de $A_v$ e $f_{p1}$, é possível 
simplificar a expressão do GBW, obtendo assim:

\begin{align}
    GBW = \frac{Gm}{2\pi \times C_{out}}    \, [Hz]
    \label{eq:GBWFinal}
\end{align}

$C_{out}$, como anteriormente mencionado, é obtido através da análise
do nó correspondente à saída. Mais concretamente, esta análise 
consiste em averiguar todas as capacidades ligadas a esse nó.

Dito isto, pela análise do circuito, obtivémos a seguinte expressão para $C_{out}$:
\begin{align}
    C_{out} = C_{gd4P} + C_{db4P} + C_{gd6N} + C_{db6N} \,[F]
    \label{eq:Cout}
\end{align}

\textcolor{red}{Colocamos alguma imagem aqui?? Tipo, dos condensadores ligados ao nó de saída??}

\subsection{Frequências do segundo e terceiro polos}

\textcolor{red}{Uma explicação melhorzinha sobre como descobrimos os polos 2 e 3}

As frequências do segundo e terceiro polos do circuito são obtidas através da análise
dos nós internos do circuito, mais concretamente, os nós que correspondem às fontes (sources)
dos transístores M4 e M6, respetivamente. A sua determinação é importante para garantir a estabilidade do
circuito em malha fechada.

As expressões para as frequências dos segundos e terceiros polos são dadas por:

\begin{align}
    F_{p2} = \dfrac{1}{2\pi \times R_{p2} \times C_{p2}} \, [Hz]
    \label{eq:fp2}
\end{align}

\begin{align}
    F_{p3} = \dfrac{1}{2\pi \times R_{p3} \times C_{p3}} \, [Hz]    
    \label{eq:fp3}
\end{align}

Para continuar mos a análise, é necessário determinar as expressões equivalentes para $R_{p2}$, $C_{p2}$, $R_{p3}$ e $C_{p3}$.
Para tal, mais uma vez, recorremos à análise do circuito nos nós correspondentes 
aos polos 2 e 3, obtendo, primeiro, as equações para a capacidade total em ambos os nós:


\begin{align}
    C_{p2} = C_{gd11} + C_{bd11} + C_{gd2a} + C_{bd2a} + C_{gs4} + C_{bs4} \, [F]  
    \label{eq:Cp2}  
\end{align}

\begin{align}
    C_{p3} = C_{gd8} + C_{bd8} + C_{gd2b} + C_{bd2b} + C_{gs6} + C_{bs6} \, [F]    
    \label{eq:Cp3}
\end{align}

E, de seguida, as resistências equivalentes vistas para cada um dos nós:

\begin{align}
    R_{p2} = \dfrac{1}{gm_4 \times (2-BE)} \, [\Omega]    
    \label{eq:Rp2}
\end{align}

\begin{align}
    R_{p3} = \dfrac{1}{gm_6 \times (2-BE)} \, [\Omega]    
    \label{eq:Rp3}
\end{align}

\textcolor{red}{Imagens aqui também??}



\subsection{OutPut Swing (OS)}

O output swing (OS) do circuito determina a gama de tensões que o circuito pode 
assumir na sua saída, sem que os transístores saíam da região de saturação. 
Este parâmetro do circuitot tem em conta as quedas de tensão 
necessárias para garantir que os transístores de saída operam 
na zona de inversão moderada, como também a margem de operação, que neste caso é 
de \SI{80}{\milli\volt}.

A expressão para o output swing do amplificador é dada por:
\begin{align}
    OS = V_{DD} - V_{11P} - V_{4P} - V_{6N} - V_{8N} - 0.08 \,[V]
    \label{eq:OS}
\end{align}


\subsection{Fator de excesso de rúido}
O fator de excesso de rúido (ENF) do circuito é um parâmetro que
indica o excesso de rúido criado por um amplificador de ganho $A_{v}$.
Um amplificador com ENF igual a 1 significa que o mesmo não acrescenta 
nenhum ruído ao sinal, ou seja, o rúido na saída não tem nenhuma contribuição
relativa ao rúido gerado pelo próprio amplificador.
Qualquer valor de ENF superior a 1 indica que o amplificador acrescenta
rúido ao sinal, que é o caso que se verifica na realidade.

O fator de rúido foi estudado para metade do circuito do CFC OTA, 
tendo em conta que os transistores em configuração de Cascode 
não contribuem com ruído. A expressão obtida para o ENF é a seguinte:
\begin{align}
    NF = 1 + \frac{ g_{m8N} + g_{m11P} }{ g_{m2aN} + g_{m2bP} }  
    \label{eq:ENF}
\end{align}

\subsection{Potência dissipada}
Para obtermos a potência dissipada pelo circuito, 
é preciso determinar a corrente total que circula no mesmo. 

Sabendo que a corrente total é dada por:
\begin{align}
    I_{total} = 3 \times I_{B} + 4 \times \frac{I_{B}}{10} \, [A]
    \label{eq:Itotal}
\end{align}

Podemos então determinar a potência dissipada pelo circuito, através da seguinte expressão:
\begin{align}
    P_{dissipada} = V_{DD} \times( 3 \times I_{B} + 4 \times \frac{I_{B}}{10})  \, [W] 
    \label{eq:Pdissipada}
\end{align}


\subsection{Figura de Mérito (FOM)}
De forma a avaliar o desempenho global do circuito, foi definida uma figura de mérito (FOM),
que tem em conta o GBW e a potência dissipada pelo circuito. A expressão para a FOM é dada por:
\begin{align}
    FoM = 1000 \cdot \frac{ GBW \cdot C_L }{ P_D }
    \label{eq:FoM}
\end{align}



\newpage