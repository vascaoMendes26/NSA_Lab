\section{Análise Teórica}

Nesta secção, vão ser mostradas todas as etapas da análise teórica para o circuito Complementary Folded-Cascode (CFC)
operational transconductance amplifier (OTA). O objetivo com esta análise é determinar os valores dos diferentes parâmetros, 
as respetivas demonstrações para efeitos de otimização, conseguindo assim cumprir os objetivos que foram definidos na tabela \ref{tab:specs}.

Para analisar teóricamente o CFC OTA, usámos como recurso uma montagem que já nos era conhecida o convencional amplificador Folded Cascode, mostrado
na Figura \ref{fig:Cascode}, o CFC parte da mesma topologia, mas tem algumas modificações levando a grandes melhorias.

\textcolor{red}{Explicaçãozinha sobre as melhorias do CFC para o normal}


\begin{figure}[htp]
    \centering
    \includegraphics[width=10cm]{Images/Fcascode.png}
    \caption{Amplificador Folded Cascode}
    \label{fig:Cascode}
\end{figure}

Ao olharmos para o circuito e como foi estudado este é diferencial, portanto podemos apenas estudar metade do circuito,
devido há sua simetria, o circuito é possível ser visto na Figura \ref{fig:Circ_Simp}.

\begin{figure}[htp]
    \centering
    \includegraphics[width=15cm]{Images/Circ_Analisado_Teorica.png}
    \caption{CFC OTA simplificado}
    \label{fig:Circ_Simp}
\end{figure}

Na Figura \ref{fig:SimplificadoMaximo} está representado o circuito simplificado, este que foi escolhido devido a ter 
o nó de saída, ficando mais simples de retirar os parâmetros e de analisar o circuito. 
Nas subsecções abaixo constaram as explicações de como foram retiradas as expressões e os valores que foram meencionados ao longo 
do relatório.

\begin{figure}[htp]
    \centering
    \includegraphics[width=15cm]{Images/Simplificadao.png}
    \caption{Circuito Simplificado}
    \label{fig:Circ_Simp}
\end{figure}


\subsection{Ganho DC}

Tal como foi estudado o ganho DC $A_v$ é obtido pelo produto da transcondutância de curto circuito $G_M$ e
a resistência de saída $R_{out}$.

\begin{equation*}
    A_v = G_M \cdot r_{out} \,\text{[V/V]}
\end{equation*}


\begin{align}
    Gm = -(g_{m2a} + g_{m2b})
\end{align}

\begin{align}
    R_{out} = r_{op} // r_{on} = \frac{1}{g_{op} + g_{on}}
\end{align}

Where $g_{op}$ and $g_{op}$ are given by:

\begin{align}
    r_{op} = \frac{1}{g_{op} + g_{on}}
\end{align}



\vspace{\baselineskip}
\begin{align}
A_v = \frac{
    \big(g_{m4P} \cdot g_{m6N}\big) \big( BE \cdot g_{m2aN} + BE \cdot g_{m2bP} \big)
}{
    BE \cdot g_{m6N} \cdot g_{ds4P} \cdot \big( g_{ds11P} + g_{ds2aN} \big)
    + BE \cdot g_{m4P} \cdot g_{ds6N} \cdot \big( g_{ds2bP} + g_{ds8N} \big)
}
\end{align}

 

\newpage