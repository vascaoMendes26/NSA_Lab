\section{Simulação}

Nesta secção é apresentada a simulação de todo o design obtido na secção anterior, todas as Figuras e valores obitdos
foram através do Cadence. 

A primeira etapa foi simular o circuito com as especificações obtidias na Tabela \ref{tab:expectedResults}, esta que como dito já está otimizada para os valores 
pretenendidos. O script python só está otimizado e o mais próximo do valor real, após serem obtidos os valores das constantes dos $g_{ds}$
devido a este facto e há utilização de valores ideias, para que não haja uma perca signficativa face aos resultados pretendidos, foi feito um varrimento para $M_{b5}$ e $M{_b6}$ para otimizar os 
resultados, os valores finais podem ser vistos na Tabela \ref{tab:transistores_dimensoes} e nas Figuras \ref{fig:bias} e \ref{fig:ota}.

\textcolor{red}{PREENCHER COM VALORES ANTIGOS}
\begin{table}[h!]
    \centering
    \caption{Dimensões finais dos Transistores implementados no Cadence}
    \label{tab:transistores_dimensoes}
    \vspace{0.10cm} % Espaço para não ficar colado ao título
    
    % Aumentar o espaçamento entre linhas para ficar legível (opcional)
    \renewcommand{\arraystretch}{1}

    \begin{tabular}{cccc}
        \toprule[1pt]
       \textbf{Transistor} & \textbf{Length (L)} & \textbf{Width (W)} & \textbf{Fingers} \\
    \midrule
    
    $M_{B1}, \, M_{B2}, \, M_{B4}$ & 800\,nm & 4.535\,$\mu$m & 1 \\
    \hline
    $M_{B3}, \, M_{B7}$ & 800\,nm & 6.53\,$\mu$m & 1 \\
    \hline
    $M_{B5}$ & 800\,nm & 380\,nm & 1 \\
    \hline
    $M_{B6}$ & 800\,nm & 800\,nm & 1 \\
    \hline
    $M_{1a}, \, M_{2a}, \, M_{1b}, \, M_{2b}$ & 1\,$\mu$m & 290\,$\mu$m & 1\\
    \hline
    $M_{3}, \, M_{4}$ & 800\,nm & 35\,$\mu$m & 1 \\
    \hline
    $M_{5}, \, M_{6}$ & 800\,nm & 38\,$\mu$m & 1 \\
    \hline
    $M_{7}, \, M_{8}$ & 800\,nm & 65.32 \,$\mu$m & 1 \\
    \hline
    $M_{9a}, \, M_{9b}$ & 800\,nm & 65.32\,$\mu$m & 1 \\
    \hline
    $M_{10}, \, M_{11}$ & 800\,nm & 45.36\,$\mu$m & 1 \\
       
        \bottomrule[1.2pt]
    \end{tabular}
\end{table}







\subsection{Ajustes Finos nos Transistores do Circuito}

Contudo e após a realização de diversas simulações, concluiu-se que o dimensionamento inicial, 
obtido através do script em Python e os ligeiros ajustes efetuados acima, 
não correspondiam ainda ao ponto de otimização máximo que gostaríamos de obter.
Embora esses valores tivessem dentro dos valores pretendidos, não refletiam o equilíbrio ideal entre Ganho, GBW e menor Potência consumida, resultando na maximização da Figura de Mérito (FoM), 
um dos claros objetivos deste trabalho.

Pra melhoarmos o desempenho, foram efetuados varrimentos paramétricos e ajustes finos nas dimensões dos transístores no Cadence, de modo a poter chegar a esse "ponto ideal de funcionamento". 
Estas iterações permitiram melhorar muito os nossos valores, conduzindo aos valores finais apresentados nas secções seguintes, os quais garantem o melhor compromisso de performance para este projeto.
Ao longo das seguintes secções iremos comparar as simulações com valores otmiziados com os valores acima obtidos,
realçando a melhoria obtida. Os valores otimizados podem ser vistos abaixo: 




\begin{figure}[h!]
    \centering
    \begin{subfigure}[b]{0.48\textwidth}
        \centering
        \includegraphics[width=\textwidth]{Images/BiasCircuit.png}
        \caption{Bias Circuit}
        \label{fig:bias}
    \end{subfigure}
    \begin{subfigure}[b]{0.48\textwidth}
        \centering
        \includegraphics[width=\textwidth]{Images/CFC_OTA.png}
        \caption{CFC OTA}
        \label{fig:ota}
    \end{subfigure}
    
    \caption{Esquemas dos circuitos: (a) Circuito de polarização e (b) OTA.}
    \label{fig:circuitos_completos}
\end{figure}


\begin{table}[h!]
    \centering
    \caption{Dimensões finais dos Transistores após otimização implementados no Cadence}
    \label{tab:transistores_dimensoes}
    \vspace{0.10cm} % Espaço para não ficar colado ao título
    
    % Aumentar o espaçamento entre linhas para ficar legível (opcional)
    \renewcommand{\arraystretch}{1}

    \begin{tabular}{cccc}
        \toprule[1pt]
       \textbf{Transistor} & \textbf{Length (L)} & \textbf{Width (W)} & \textbf{Fingers} \\
    \midrule
    
    $M_{B1}, \, M_{B2}, \, M_{B4}$ & 800\,nm & 4.535\,$\mu$m & 1 \\
    \hline
    $M_{B3}, \, M_{B7}$ & 800\,nm & 6.53\,$\mu$m & 1 \\
    \hline
    $M_{B5}$ & 800\,nm & 380\,nm & 1 \\
    \hline
    $M_{B6}$ & 800\,nm & 800\,nm & 1 \\
    \hline
    $M_{1a}, \, M_{2a}, \, M_{1b}, \, M_{2b}$ & 1\,$\mu$m & 290\,$\mu$m & 1\\
    \hline
    $M_{3}, \, M_{4}$ & 800\,nm & 35\,$\mu$m & 1 \\
    \hline
    $M_{5}, \, M_{6}$ & 800\,nm & 38\,$\mu$m & 1 \\
    \hline
    $M_{7}, \, M_{8}$ & 800\,nm & 65.32 \,$\mu$m & 1 \\
    \hline
    $M_{9a}, \, M_{9b}$ & 800\,nm & 65.32\,$\mu$m & 1 \\
    \hline
    $M_{10}, \, M_{11}$ & 800\,nm & 45.36\,$\mu$m & 1 \\
       
        \bottomrule[1.2pt]
    \end{tabular}
\end{table}

Após terem sido obtidos os tamanhos de cada transistor, foi criado o bloco completo, que será usado nas simulações abaixo. Este circuito
é possível ser visto na Figura \ref{fig:TudoJunto}.

\begin{figure}[htp]
    \centering
    \includegraphics[width=15cm]{Images/TudoJunto.png}
    \caption{Símbolo do Circuito no Cadence}
    \label{fig:TudoJunto}
\end{figure}

\newpage

\subsection{Simulação DC}

A simulação DC foi realizada com o objetivo de verificar a polarização do circuito, estando os resultados obtidos ilustrados na Figuras \ref{fig:BiasOP} e \ref{fig:otaOP}.
Estes resultados encontram-se igualmente compilados nas Tabelas \ref{tab:dc_bias} e \ref{tab:dc_main}. 

A análise do ponto de funcionamento mostra que no geral os transístores operam na região de inversão moderada, esta região de operação é adequada, não comprometemetendo o desempenho.
A discrepância no valor das correntes do circuito princcipal 
deve-se à não idealidade dos transístores, sendo as correntes influenciadas pelas dimensões dos dispositivos e pelo circuito de polarização. 

Embora este facto não represente um problema crítico visto que o circuito funciona conforme o esperado e as 
correntes apresentam valores razoáveis para o seu correto funcionamento, não atingido os valores máximos desta topologia.

Este desempenho e como referido acima foi aperfeiçoado através de uma otimização, esta que teve que ser realizada analisada com um pequeno detalhe,
respeitar as restrições mencionadas na Tabela \ref{tab:specs}. Estes valores obtidos podem ser vistos abaixo: 



\begin{figure}[h!]
    \centering
    \begin{subfigure}[b]{0.49\textwidth}
        \centering
        \includegraphics[width=\textwidth]{Images/BiasOP.png}
        \caption{Simulação DC para circuto de Bias}
        \label{fig:BiasOP}
    \end{subfigure}
    \begin{subfigure}[b]{0.49\textwidth}
        \centering
        \includegraphics[width=\textwidth]{Images/OTAOP.png}
        \caption{Simulação DC para circuto principal}
        \label{fig:otaOP}
    \end{subfigure}
    
    \caption{Resultados da Simulação DC para ambos os circuitos}
    \label{fig:circuitos_completos}
\end{figure}





\begin{table}[h!]
    \centering
    \caption{Comparação dos Resultados DC: Valores Iniciais vs. Otimizados}
    \label{tab:dc_bias_compare}
    \vspace{0.2cm}
    \renewcommand{\arraystretch}{1.1} % Um pouco mais de espaço

    % --- Tabela da Esquerda (Valores Antigos) ---
    \begin{minipage}[t]{0.48\textwidth}
        \centering
        \subcaption*{Valores Iniciais (Sem otimização)} % Requer package subcaption ou apenas texto em negrito
        % Se der erro no \subcaption*, usa: \textbf{Valores Iniciais} \\ \vspace{0.1cm}
        
        \resizebox{\textwidth}{!}{ % Ajusta a tabela à largura da minipage
        \begin{tabular}{ccc}
            \toprule[1pt]
            \textbf{Trans.} & \boldmath{$V_{DSat}$} & \boldmath{$I_{D}$} \\
            \midrule
            $M_{B1}$ & $-155.7 \, \text{mV}$ & $-8 \, \mu\text{A}$ \\ \hline
            $M_{B2}$ & $-155.7 \, \text{mV}$ & $-8.9 \, \mu\text{A}$ \\ \hline
            $M_{B3}$ & $88.8 \, \text{mV}$   & $8.9 \, \mu\text{A}$ \\ \hline
            $M_{B4}$ & $-155.7 \, \text{mV}$ & $-8.3 \, \mu\text{A}$ \\ \hline
            $M_{B5}$ & $297.5 \, \text{mV}$  & $8.3 \, \mu\text{A}$ \\ \hline
            $M_{B6}$ & $-346.9 \, \text{mV}$ & $-9.2 \, \mu\text{A}$ \\ \hline
            $M_{B7}$ & $89.2 \, \text{mV}$   & $9.2 \, \mu\text{A}$ \\ 
            \bottomrule[1pt]
        \end{tabular}
        }
    \end{minipage}
    \hfill % Empurra as tabelas para as extremidades opostas
    % --- Tabela da Direita (Valores Atuais/Otimizados) ---
    \begin{minipage}[t]{0.48\textwidth}
        \centering
        \subcaption*{Valores Finais (Otimizados)} 
        % Alternativa: \textbf{Valores Finais} \\ \vspace{0.1cm}

        \resizebox{\textwidth}{!}{ % Ajusta a tabela à largura da minipage
        \begin{tabular}{ccc}
            \toprule[1pt]
            \textbf{Trans.} & \boldmath{$V_{DSat}$} & \boldmath{$I_{D}$} \\
            \midrule
            $M_{B1}$ & $-155.7 \, \text{mV}$ & $-8 \, \mu\text{A}$ \\ \hline
            $M_{B2}$ & $-155.7 \, \text{mV}$ & $-8.9 \, \mu\text{A}$ \\ \hline
            $M_{B3}$ & $88.8 \, \text{mV}$   & $8.9 \, \mu\text{A}$ \\ \hline
            $M_{B4}$ & $-155.7 \, \text{mV}$ & $-8.3 \, \mu\text{A}$ \\ \hline
            $M_{B5}$ & $297.5 \, \text{mV}$  & $8.3 \, \mu\text{A}$ \\ \hline
            $M_{B6}$ & $-346.9 \, \text{mV}$ & $-9.2 \, \mu\text{A}$ \\ \hline
            $M_{B7}$ & $89.2 \, \text{mV}$   & $9.2 \, \mu\text{A}$ \\ 
            \bottomrule[1pt]
        \end{tabular}
        }
    \end{minipage}
\end{table}

\begin{table}[h!]
    \centering
    \caption{Comparação dos Resultados DC do Circuito Principal: Inicial vs. Otimizado}
    \label{tab:dc_main_compare}
    
    % Ajustes de tamanho e espaçamento
    \small % Diminui a fonte (pode usar \footnotesize se quiser menor)
    \renewcommand{\arraystretch}{0.85} % Reduz a altura entre as linhas
    \setlength{\tabcolsep}{4pt} % Ajusta o espaço entre colunas

    \begin{tabular}{c|cc|cc}
        \toprule[1pt]
        & \multicolumn{2}{c|}{\textbf{Inicial(Sem otimização)}} & \multicolumn{2}{c}{\textbf{Final (Otimizados para Sim.)}} \\
        \textbf{Trans.} & \boldmath{$V_{DSat}$} & \boldmath{$I_{D}$} & \boldmath{$V_{DSat}$} & \boldmath{$I_{D}$} \\
        \midrule
        
        $M_{1a}$ & 0 & 0 & $58.3 \, \text{mV}$ & $42.52 \, \mu\text{A}$ \\ \hline
        $M_{1b}$ & 0 & 0 & $-78.56 \, \text{mV}$ & $-36.57 \, \mu\text{A}$ \\ \hline
        $M_{2a}$ & 0 & 0 & $58.24 \, \text{mV}$ & $42.39 \, \mu\text{A}$ \\ \hline
        $M_{2b}$ & 0 & 0 & $-78.69 \, \text{mV}$ & $-36.66 \, \mu\text{A}$ \\ \hline
        $M_{3}$  & 0 & 0 & $-93.14 \, \text{mV}$ & \textcolor{red}{$-31.85 \, \mu\text{A}$} \\ \hline
        $M_{4}$  & 0 & 0 & $-96.53 \, \text{mV}$ & \textcolor{red}{$-31.85 \, \mu\text{A}$} \\ \hline
        $M_{5}$  & 0 & 0 & $82.2 \, \text{mV}$ & \textcolor{red}{$31.85 \, \mu\text{A}$} \\ \hline
        $M_{6}$  & 0 & 0 & $80.16 \, \text{mV}$ & \textcolor{red}{$31.85 \, \mu\text{A}$} \\ \hline
        $M_{7}$  & 0 & 0 & $81.48 \, \text{mV}$ & \textcolor{red}{$68.42 \, \mu\text{A}$} \\ \hline
        $M_{8}$  & 0 & 0 & $81.49 \, \text{mV}$ & \textcolor{red}{$68.51 \, \mu\text{A}$} \\ \hline
        $M_{9a}$ & 0 & 0 & $87.77 \, \text{mV}$ & $84.92 \, \mu\text{A}$ \\ \hline
        $M_{9b}$ & 0 & 0 & $-156.6 \, \text{mV}$ & $-73.23 \, \mu\text{A}$ \\ \hline
        $M_{10}$ & 0 & 0 & $-156.6 \, \text{mV}$ & $-74.38 \, \mu\text{A}$ \\ \hline
        $M_{11}$ & 0 & 0 & $-156.6 \, \text{mV}$ & $-74.24 \, \mu\text{A}$ \\
        
        \bottomrule[1pt]
    \end{tabular}
\end{table}

\newpage
\subsection{Simulação AC}

\begin{figure}[htp]
    \centering
    \includegraphics[width=14cm]{Images/AC_Sim.png}
    \caption{Diagrama de bode do circuito}
    \label{fig:Sim_AC}
\end{figure}



\begin{table}[ht]
    \centering
    \caption{AC simulation results}
    \label{tab:ac_results}
    \vspace{0.2cm}
    \centering
    % Alterado de {cc} para {lcc} (Left, Center, Center) para melhor alinhamento
    \begin{tabular}{lcc} 
        \toprule[1.2pt]
        \textbf{Parâmetro} & \textbf{Não Otimizados} & \textbf{Após Otimização} \\
        \midrule
        
        Ganho DC & $... \, \text{dB}$ & $62.89 \, \text{dB}$ \\
        \hline
        GBW & $... \, \text{MHz}$ & $58.34 \, \text{MHz}$ \\
        \hline
        Margem de Fase & $...^{\circ}$ & $65.42^{\circ}$ \\
        \hline
        Output Swing & $... \, \text{mV}$ & $686.1 \, \text{mV}$ \\
        \hline
        Potência Consumida & $... \, \mu\text{W}$ & $306.4 \, \mu\text{W}$ \\
        \hline
        Excess-Noise Factor & $...$ & $2.145$ \\
        \hline
        Figure of Merit & $... \, \text{MHz} \cdot \text{pF/mW}$ & $952.11 \, \text{MHz} \cdot \text{pF/mW}$ \\
        
        \bottomrule[1.2pt]
    \end{tabular}
\end{table}

\textcolor{red}{Justifcar os valores acima obtidos com os as subsecções anteriores e mostrar que valeu apena o sweep}

%\subsection{Layout}



%\subsection{Melhorias no Circuito}

\newpage

\subsection{Comparação com o Folded-Cascode OTA}

Com os resultados obtidos para o CFC OTA, torna-se pertinente efetuar uma comparação da sua Figura de Mérito (FoM) com a do \textit{Folded-Cascode} OTA, cuja topologia se encontra representada na Figura \ref{fig:Cascode} e que foi análisado no trabalho laboratorial anterior.

Para além das evidentes diferenças entre as duas topologias, destaca-se a diferença da capacidade de carga ($C_{load}$) imposta era de $8\,\text{pF}$. 
Esta carga capacitiva elevada obrigou, na altura, a um aumento substancial da corrente de polarização para garantir a estabilidade e a largura de banda exigidas. 
Consequentemente, embora se tenha alcançado um GBW de aproximadamente $250\,\text{MHz}$, o valor de potência 
foi cerca de 4 vezes maior devido ao alto aumento de corrente, resultando numa FoM de cerca de 5500, muito maior do que a obtida no CFC, esta que como foi visto 
fora obtido o valor de 900.

Esta comparação evidencia o impacto direto que o dimensionamento da carga ($C_{load}$) exerce sobre o consumo de potência. 



\newpage
