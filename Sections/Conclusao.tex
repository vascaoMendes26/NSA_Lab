\section{Análise de Resultados e Conclusão}

Nesta secção, iremos analisar os resultados obtidos na simulação do OTA, comparando com valores teóricos e apontar alguns pontos de melhoria futuros. 

Começando por analisar os resultados obtidos no script python com os valores simulados antes da otimização, verificamos que o dimensionamento foi efetuado corretamente, 
mas que  devido a uso de valores ideais, alguns valores para os tamanhos dos transistores fugiram um pouco do que era esperado. Após ter sido efetuada a otimização e falando agora
desses resultados, é possível ver que os valores continuam a rondar muito perto dos valores do python, justificando a fidelidade dos resultados obtidos no script.

Passando para a simulação DC, os resultados também eram os esperados e obtivemos valores de $V_{dsats}$ e de $I_D$ muito próximos dos definidos no python o que considerámos
um resultado muito bom, estando cada vez mais perto do valor teórico e mais perto dos objetivos definidos anteriormente.   

Os resultados da simulação AC, dão-nos a confirmação de que todo o dimensionamento para trás foi bem efetuado, onde foi garantido a margem de fase mínima de $60^\circ$, obtendo um 
valor de 2º polo bastante longe do polo dominante garantido assim uma boa estabilidade.
Na Tabela \ref{tab:ac_results}, é possível ver o resultado obtido para os parâmetros definidos anteriormente e a comparação com os valores que tinham sido obtidos antes das iterações, 
esta que tornou possível entender a importância de otimizar os valores e verificar se todos os transistores estão a funcionar no ponto ideal, obtendo melhores valores em todos os parâmetros
com destaque à figura de mérito. Os resultados finais revelam-se novamente valores de sucesso, cumprindo os objetivos do trabalho. 

Por fim, analisando o Layout do circuito, percebemos a importância dos fingers. A sua utilização revelou-se obrigatória, pois aumentar o número de fingers permitiu-nos 
dividir transístores de grandes dimensões em múltiplos segmentos menores, conseguindo chegar à Figura \ref{fig:Layout}.

Os objetivos deste trabalho foram cumpridos com sucesso, onde conseguímos fazer o design, otimizar e simular um amplificador CFC OTA.




