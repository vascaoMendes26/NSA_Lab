\section{Dimensionamento}


O processo de dimensionamento do circuito consiste em, com base 
nos resultados obtidos da análise teórica, optimizar os parâmetros de interesse
de modo a cumprir os requisitos especificados, tendo sempre em conta restrições impostas.
De acordo a análise teórica, foi desenvolvido um script 
em \textit{Python} que permite não só calcular os parâmetros de interesse do circuito
em função das características dos diversos transístores, como também 
calculcar a razão de tamanho de cada transístor.

Começou-se por definir para cada transístor a sua tensão de saturação como também o seu 
comprimento. Nesta fase inicial, considerou-se para todos os transístores um comprimento
de $L = \SI{0.6}{\micro\meter}$ e uma tensão de saturação de $V_{DS,sat} = \SI{0.1}{\volt}$.
Daí para a frente, com base nos gráficos obtidos em função dos valores anteriormente mencionados,
procurámos encontrar as melhores características para cada transístor de modo 
a optimizar o desempenho do circuito.


\subsection{Análise ganho DC}
Como já foi anteriormente referido, o ganho em DC do CFC 
tem que estar próximo de $62dB$. Para atingir este valor, é preciso
averiguar quais os transístores que influenciam este parâmetro \newline\newline


\begin{table}[h!]
    \centering
    % --- ESTA LINHA É QUE FAZ APARECER NO ÍNDICE ---
    \caption{características dos Transístores que afetam o ganho DC} 
    \label{tab:gainParameters} 
    % -----------------------------------------------
    \vspace{0.2cm}
    \setlength{\tabcolsep}{25pt}  % aumenta o espaço entre colunas

    \begin{tabular}{ccc}
        \toprule[1.2pt]
        \textbf{Transístor} & \textbf{Comprimento} & \textbf{Tensão de Saturação} \\
        \hline
        $M_{2aN}$ & Afeta & Afeta \\[0.25cm]
    
        $M_{2bP}$ & Afeta & Afeta \\[0.25cm]
        
        $M_{4P}$ & Afeta & Afeta \\[0.25cm]
        
        $M_{6N}$ & Afeta & Afeta \\[0.25cm]
        
        $M_{8N}$ & Afeta & Não Afeta \\[0.25cm]
        
        $M_{11P}$ & Afeta & Não Afeta \\[0.25cm]
        \bottomrule[1.5pt] 
    \end{tabular}
\end{table}

\newpage
Com base na tabela~\ref{tab:gainParameters}, podemos averiguar 
quais as características dos diferentes transístores que influenciam o ganho DC do circuito.
Contudo, a informação desta tabela não é suficiente para dimensionar os componentes do mesmo.

Para tal, no script em \textit{Python} foram criados gráficos que relacionam o ganho DC com
a tensão de saturação e com o comprimento dos transístores, permitindo uma melhor perspectiva
de como cada característica influencia o ganho DC.

\begin{figure}[htp]
    \centering
    \includegraphics[width=15cm]{Images/GainDCGraph.png}
    \caption{Ganho DC em função de $V_{DSsat}$ e $L$}
    \label{fig:GainDCGraph}
\end{figure}

Na figura~\ref{fig:GainDCGraph} podemos observar 
a relação do ganho DC com a tensão de saturação e com o comprimento dos transístores.

Nesta mesma figura, podemos averiguar a importância das duas características dos
diversos transístores no ganho DC do circuito.

\subsection{Análise do produto do Ganho com Largura de banda (GBW)}
Para o GBW, neste projeto foi nos requisitado um valor aproximado de 50Mhz.
Tal como no ganho DC, aqui também foi necessário averiguar quais os transístores
que influenciam este parâmetro. 

Para construirmos esta tabela, tivemos em atenção a equação~\ref{eq:gm}, 
a equação~\ref{eq:GBWFinal} e a equação~\ref{eq:Cout}. 
Estas equações são fundamentais para
determinar quais os transístores que influenciam o GBW.

\begin{table}[h!]
    \centering
    % --- ESTA LINHA É QUE FAZ APARECER NO ÍNDICE ---
    \caption{características dos Transístores que afetam o GBW} 
    \label{tab:gainParameters} 
    % -----------------------------------------------
    \vspace{0.2cm}
    \setlength{\tabcolsep}{25pt}  % aumenta o espaço entre colunas

    \begin{tabular}{ccc}
        \toprule[1.2pt]
        \textbf{Transístor} & \textbf{Comprimento} & \textbf{Tensão de Saturação} \\
        \hline
        $M_{2aN}$ & Não Afeta & Afeta \\[0.25cm]
    
        $M_{2bP}$ & Não Afeta & Afeta \\[0.25cm]
        
        $M_{4P}$ & Afeta & Afeta \\[0.25cm]
        
        $M_{6N}$ & Afeta & Não Afeta \\[0.25cm]
        
        \bottomrule[1.5pt] 
    \end{tabular}
\end{table}

Também para o GBW, foram criados gráficos que relacionam o GBW com
a tensão de saturação e com o comprimento dos transístores, permitindo
uma melhor perspectiva para o dimensionamento dos componentes do circuito.


\begin{figure}[htp]
    \centering
    \includegraphics[width=15cm]{Images/GBWGraph.png}
    \caption{Ganho DC em função de $V_{DSsat}$ e $L$}
    \label{fig:GBWGraph}
\end{figure}

\newpage
Na figura~\ref{fig:GBWGraph} podemos observar 
a relação do GBW com a tensão de saturação e com o comprimento dos transístores.

Nesta mesma figura, podemos averiguar a importância das duas características dos
diversos transístores no GBW do circuito.

\subsection{Análise da Frequência dos polos}
Neste trabalho é requisito que a frequência do segundo polo seja superior a 200MHz e 
a frequência do terceiro polo seja superior a 500MHz, de forma a garantir
estabilidade numa configuração em malha-fechada de ganho unitário.

Para tal, aqui também foram obtidos os gráficos correspondentes, como nas secções 
anteriores.
\newpage
\begin{figure}[htp]
    \centering
    \includegraphics[width=15cm]{Images/fp2Graph.png}
    \caption{Frequência do segundo polo em função de $V_{DSsat}$ e $L$}
    \label{fig:fp2Graph}
\end{figure}


Pela análise da figura~\ref{fig:fp2Graph}, podemos observar
a relação da frequência do segundo polo com a tensão de saturação e com o comprimento dos 
transístores. Aqui podemos notar que o transístor $M_{4P}$ é o que mais influencia
o comportamento da frequência do segundo polo.

\begin{figure}[htp]
    \centering
    \includegraphics[width=15cm]{Images/fp3Graph.png}
    \caption{Frequência do terceiro polo em função de $V_{DSsat}$ e $L$}
    \label{fig:fp3Graph}
\end{figure}

Pela análise da figura~\ref{fig:fp3Graph}, podemos observar
a relação da frequência do terceiro polo com a tensão de saturação e com o comprimento dos
transístores. Aqui podemos notar que a tensão de saturação do transístor $M_{4P}$ é a
que mais influencia o comportamento da frequência do terceiro polo, mas como 
também o comprimento do transístor $M_{6N}$ tem uma influência significativa.


\subsection{Análise do Output Swing(OS)}
Pela equação~\ref{eq:OS}, o valor associado ao output swing varia
em função dos valores de $V_{DS,sat}$ dos transístores $M_{4P}$ ,$M_{6N}$, $M_{8N}$ e $M_{11P}$.

Desta forma, não houve necessidade de criar gráficos adicionais, uma vez
que existe uma relação direta entre o output swing e a tensão de saturação dos diversos
transístores mencionados.


\subsection{Análise do Fator de excesso de rúido}
O fator de excesso de rúido (ENF) do circuito é 
um parâmetro que encontra-se afetado pelos valores de $V_{DS,sat}$ dos 
transístores $M_{2aN}$, $M_{2bP}$, $M_{8N}$ e $M_{11P}$.

Tendo isso em conta, o seguinte gráfico foi obtido, relacionando o ENF com a tensão de saturação
desses transistores já mencionados.

\begin{figure}[htp]
    \centering
    \includegraphics[width=15cm]{Images/NFGraph.png}
    \caption{Fator de excesso de rúido em função de $V_{DSsat}$}
    \label{fig:NFGraph}
\end{figure}

A figura~\ref{fig:NFGraph} mostra a relação anteriormente referida.

\subsection{Análise da potência dissipada}
Pela equação~\ref{eq:Pdissipada}, a corrente de Bias é o único parâmetro que pode ser 
variável de forma a influenciar a potência dissipada no circuito.

No caso, numa primeira fase do dimensionamento, considerou-se um valor de $I_{B} = \SI{100}{\micro\ampere}$. 
Numa fase posterior, viu-se a necessidade de reduzir este valor para $I_{B} = \SI{75}{\micro\ampere}$, de forma
a melhorar outros parâmetros, como o GBW, dependentes desta corrente.


\subsection{Dimensionamento Final}
Antes de proceder ao dimensionamento final do circuito, 
é importante estabelecer relações entre os diversos transistores presentes no circuito.

Como mencionado na análise teórica, a análise do circuito
foi feita considerando uma das metades do mesmo, sendo a metade analisada
correspondente aos transístores de saída do CFC.

Daí a importância de estabelecer as relações entre os transistores.
As seguintes equações representam essas relações:

\begin{align}
    M_{2aN} = M_{1aN}
    \label{eq:ratio1}
\end{align}
\begin{align}   
    M_{2bP} = M_{1bP}
    \label{eq:ratio2} 
\end{align}
\begin{align}
    M_{4P} = M_{3P}
    \label{eq:ratio3} 
\end{align}
\begin{align}
    M_{6N} = M_{5N}
    \label{eq:ratio4}
\end{align}
\begin{align}
    M_{8N} = M_{7N}
    \label{eq:ratio5}
\end{align}
\begin{align}
    M_{11P} = M_{10P}
    \label{eq:ratio6}
\end{align}

O dimensionamento dos transístores $M_{9aN}$ e $M_{9bP}$ foi feito 
de formar a ajustar os parâmetros de interesse do circuito, daí 
a sua ausência nas equações anteriores.

Também foi do nosso interesse dimensionar o circuito Bias. Para tal,
aqui também foram estabelecidas relações entre os transístores do circuito Bias com os 
transístores do CFC OTA, como se pode observar nas seguintes equações:

\begin{align}
    M_{B1P} = M_{B2P} = M_{4BP} = M_{9bP}
    \label{eq:bias1}
\end{align}
\begin{align}
    M_{B3N} = M_{B7N} = M_{9aN}
    \label{eq:bias2}
\end{align}
\begin{align}
    M_{B5N} = M_{6N} 
    \label{eq:bias3}
\end{align}
\begin{align}
    M_{B6P} = M_{4P} 
    \label{eq:bias4}
\end{align}

De notar que, para o circuito Bias, considerou-se uma corrente 
10 vezes menor que a corrente de Bias do CFC OTA, ou seja, $I_{Bias} = \frac{I_{B}}{10}$.

Após a análise de todos os parâmetros de interesse do circuito e com o
auxílio do script desenvolvido em \textit{Python}, 
chegámos a um dimensionamento final para o circuito, 
cujo resumo se encontra na tabela~\ref{tab:finalSizing}. 


\begin{table}[h!]
    \centering
    % --- ESTA LINHA É QUE FAZ APARECER NO ÍNDICE ---
    \caption{Dimensionamento Final dos Transístores(Teórico)} 
    \label{tab:finalSizing} 
    % -----------------------------------------------
    \vspace{0.2cm}
    \setlength{\tabcolsep}{25pt}  % aumenta o espaço entre colunas

    \begin{tabular}{cccc}
        \toprule[1.2pt]
        \textbf{Transístor} & \textbf{Comprimento} & \textbf{Largura} & \textbf{VDSsat}\\
        \hline
        $M_{2aN}$ & $\SI{100}{\nano\meter}$ & $\SI{90.30}{\micro\meter}$ & $\SI{55}{\milli\volt}$\\[0.25cm]

        $M_{2bP}$ & $\SI{100}{\nano\meter}$ & $\SI{90.30}{\micro\meter}$ & $\SI{55}{\milli\volt}$\\[0.25cm]

        $M_{4P}$ & $\SI{800}{\nano\meter}$ & $\SI{26.03}{\micro\meter}$ & $\SI{97}{\milli\volt}$\\[0.25cm]
        
        $M_{6N}$ & $\SI{800}{\nano\meter}$ & $\SI{20.50}{\micro\meter}$ & $\SI{100}{\milli\volt}$\\[0.25cm]

        $M_{8N}$ & $\SI{800}{\nano\meter}$ & $\SI{40.49}{\micro\meter}$ & $\SI{110}{\milli\volt}$\\[0.25cm]

        $M_{11P}$ & $\SI{800}{\nano\meter}$ & $\SI{21.77}{\micro\meter}$ & $\SI{150}{\milli\volt}$\\[0.25cm]

        $M_{9aN}$ & $\SI{800}{\nano\meter}$ & $\SI{34.02}{\micro\meter}$ & $\SI{120}{\milli\volt}$\\[0.25cm]

        $M_{9bP}$ & $\SI{800}{\nano\meter}$ & $\SI{57.16}{\micro\meter}$ & $\SI{120}{\milli\volt}$\\[0.25cm]
        
        \bottomrule[1.5pt] 
    \end{tabular}
\end{table}

\newpage
Uma tabela semelhante foi também criada para o circuito Bias,
cujo resumo se encontra na tabela~\ref{tab:finalSizingBias}.

\begin{table}[h!]
    \centering
    % --- ESTA LINHA É QUE FAZ APARECER NO ÍNDICE ---
    \caption{Dimensionamento Final dos Transístores do circuito Bias (Teórico)} 
    \label{tab:finalSizingBias} 
    % -----------------------------------------------
    \vspace{0.2cm}
    \setlength{\tabcolsep}{25pt}  % aumenta o espaço entre colunas

    \begin{tabular}{ccc}
        \toprule[1.2pt]
        \textbf{Transístor} & \textbf{Comprimento} & \textbf{Largura}\\
        \hline
        $M_{B1P}$ & $\SI{800}{\nano\meter}$ & $\SI{6.04}{\micro\meter}$\\[0.25cm]

        $M_{B2P}$ & $\SI{800}{\nano\meter}$ & $\SI{6.04}{\micro\meter}$\\[0.25cm]

        $M_{B3N}$ & $\SI{800}{\nano\meter}$ & $\SI{3.38}{\micro\meter}$\\[0.25cm]
        
        $M_{4BP}$ & $\SI{800}{\nano\meter}$ & $\SI{6.04}{\micro\meter}$\\[0.25cm]

        $M_{B5N}$ & $\SI{800}{\nano\meter}$ & $\SI{4.86}{\micro\meter}$\\[0.25cm]

        $M_{B6P}$ & $\SI{800}{\nano\meter}$ & $\SI{9.24}{\micro\meter}$\\[0.25cm]

        $M_{B7N}$ & $\SI{800}{\nano\meter}$ & $\SI{3.38}{\micro\meter}$\\[0.25cm]
        
        \bottomrule[1.5pt] 
    \end{tabular}
\end{table}

Dados os valores da tabela~\ref{tab:finalSizing}, o resultado esperado para 
a simulação do circuito é dado na seguinte tabela~\ref{tab:expectedResults}.


\begin{table}[h!]
    \centering
    % --- ESTA LINHA É QUE FAZ APARECER NO ÍNDICE ---
    \caption{Resultados Esperados na simulação} 
    \label{tab:expectedResults} 
    % -----------------------------------------------
    \vspace{0.2cm}

    \begin{tabular}{cc}
        \toprule[1.2pt] 
        \textbf{Parâmetro} & \textbf{Resultado} \\ 
        \midrule 
        Ganho & 62.365  \text{dB} \\ 
        \hline
        GBW  & 69.007 \text{MHz} \\ 
        \hline
        $f_{p2}$ & 348.900 \text{MHz} \\ 
        \hline
        $f_{p3}$ &  500.520 \text{MHz} \\ 
        \hline
        Output Swing (OS)  & 663.000 \text{mV} \\ 
        \hline
        Fator de excesso de ruído & 1.87 \\ 
        \hline
        Potência dissipada & 306.000 \text{$\mu$W} \\
        \hline
        Figura de Mérito (FOM) & $1128\,s/{V^{-2}}$ \\
        \bottomrule[1.5pt] 
    \end{tabular}
\end{table}



\newpage