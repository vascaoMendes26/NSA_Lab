\section{Simulação}

Nesta secção é apresentada a simulação de todo o design obtido na secção anterior, todas as Figuras e valores obtidos
foram através do Cadence. 

A primeira etapa foi simular o circuito com as especificações obtidas na Tabela \ref{tab:expectedResults}, esta que como dito já está otimizada para os valores 
pretendidos. O script python só está otimizado e o mais próximo do valor real, após serem obtidos os valores das constantes dos $g_{ds}$
devido a isto e ao facto de utilizarmos valores ideias, para que não haja uma perda signficativa face aos resultados pretendidos, foi feito um varrimento para $M_{b5}$ e $M_{b6}$ para otimizar os 
resultados, os valores finais podem ser vistos na Tabela \ref{tab:transistores_dimensoes} e nas Figuras \ref{fig:bias} e \ref{fig:ota}.

\begin{table}[h!]
    \centering
    \caption{Dimensões finais dos Transistores implementados no Cadence}
    \label{tab:transistores_dimensoes}
    \vspace{0.10cm} % Espaço para não ficar colado ao título
    
    % Aumentar o espaçamento entre linhas para ficar legível (opcional)
    \renewcommand{\arraystretch}{1}

    \begin{tabular}{cccc}
        \toprule[1pt]
       \textbf{Transistor} & \textbf{Length (L)} & \textbf{Width (W)} & \textbf{Fingers} \\
    \midrule
    
    $M_{B1}, \, M_{B2}, \, M_{B4}$ & 800\,nm & 4.535\,$\mu$m & 1 \\
    \hline
    $M_{B3}, \, M_{B7}$ & 800\,nm & 6.53\,$\mu$m & 1 \\
    \hline
    $M_{B5}$ & 800\,nm & 670\,nm & 1 \\
    \hline
    $M_{B6}$ & 800\,nm & 740\,nm & 1 \\
    \hline
    $M_{1a}, \, M_{2a}, \, M_{1b}, \, M_{2b}$ & 1\,$\mu$m & 145.68\,$\mu$m & 1\\
    \hline
    $M_{3}, \, M_{4}$ & 800\,nm & 66.65\,$\mu$m & 1 \\
    \hline
    $M_{5}, \, M_{6}$ & 800\,nm & 32.66\,$\mu$m & 1 \\
    \hline
    $M_{7}, \, M_{8}$ & 800\,nm & 65.32 \,$\mu$m & 1 \\
    \hline
    $M_{9a}, \, M_{9b}$ & 800\,nm & 65.32\,$\mu$m & 1 \\
    \hline
    $M_{10}, \, M_{11}$ & 800\,nm & 45.36\,$\mu$m & 1 \\
       
        \bottomrule[1.2pt]
    \end{tabular}
\end{table}







\subsection{Ajustes Finos nos Transistores do Circuito}

Contudo e após a realização de diversas simulações, concluiu-se que o dimensionamento inicial, 
obtido através do script em Python e os ligeiros ajustes efetuados acima, 
não correspondiam ainda ao ponto de otimização máximo que gostaríamos de obter.
Embora esses valores estivessem dentro dos valores pretendidos, não refletiam o equilíbrio ideal entre Ganho, GBW e menor Potência consumida, resultando na maximização da Figura de Mérito (FoM), 
um dos claros objetivos deste trabalho.

Para melhorarmos o desempenho, foram efetuados varrimentos paramétricos e ajustes finos nas dimensões dos transístores no Cadence, de modo a poder chegar a esse "ponto ideal de funcionamento". 
Estas iterações permitiram melhorar muito os nossos valores, conduzindo aos valores finais apresentados nas secções seguintes, os quais garantem o melhor compromisso de performance para este projeto.
Ao longo das seguintes secções iremos comparar as simulações com valores otimizados com os valores acima obtidos,
realçando a melhoria obtida. Os valores otimizados podem ser vistos abaixo: 




\begin{figure}[h!]
    \centering
    \begin{subfigure}[b]{0.48\textwidth}
        \centering
        \includegraphics[width=\textwidth]{Images/BiasCircuit.png}
        \caption{Bias Circuit}
        \label{fig:bias}
    \end{subfigure}
    \begin{subfigure}[b]{0.48\textwidth}
        \centering
        \includegraphics[width=\textwidth]{Images/CFC_OTA.png}
        \caption{CFC OTA}
        \label{fig:ota}
    \end{subfigure}
    
    \caption{Esquemas dos circuitos: (a) Circuito de polarização e (b) OTA.}
    \label{fig:circuitos_completos}
\end{figure}


\begin{table}[h!]
    \centering
    \caption{Dimensões finais dos Transistores após otimização implementados no Cadence}
    \label{tab:transistores_dimensoes}
    \vspace{0.10cm} % Espaço para não ficar colado ao título
    
    % Aumentar o espaçamento entre linhas para ficar legível (opcional)
    \renewcommand{\arraystretch}{1}

    \begin{tabular}{cccc}
        \toprule[1pt]
       \textbf{Transistor} & \textbf{Length (L)} & \textbf{Width (W)} & \textbf{Fingers} \\
    \midrule
    
    $M_{B1}, \, M_{B2}, \, M_{B4}$ & 800\,nm & 4.535\,$\mu$m & 1 \\
    \hline
    $M_{B3}, \, M_{B7}$ & 800\,nm & 6.53\,$\mu$m & 1 \\
    \hline
    $M_{B5}$ & 800\,nm & 380\,nm & 1 \\
    \hline
    $M_{B6}$ & 800\,nm & 800\,nm & 1 \\
    \hline
    $M_{1a}, \, M_{2a}, \, M_{1b}, \, M_{2b}$ & 1\,$\mu$m & 290\,$\mu$m & 1\\
    \hline
    $M_{3}, \, M_{4}$ & 800\,nm & 35\,$\mu$m & 1 \\
    \hline
    $M_{5}, \, M_{6}$ & 800\,nm & 38\,$\mu$m & 1 \\
    \hline
    $M_{7}, \, M_{8}$ & 800\,nm & 65.32 \,$\mu$m & 1 \\
    \hline
    $M_{9a}, \, M_{9b}$ & 800\,nm & 65.32\,$\mu$m & 1 \\
    \hline
    $M_{10}, \, M_{11}$ & 800\,nm & 45.36\,$\mu$m & 1 \\
       
        \bottomrule[1.2pt]
    \end{tabular}
\end{table}

Após terem sido obtidos os tamanhos de cada transistor, foi criado o bloco completo, que será usado nas simulações abaixo. Este circuito
é possível ser visto na Figura \ref{fig:TudoJunto}.

\begin{figure}[htp]
    \centering
    \includegraphics[width=15cm]{Images/TudoJunto.png}
    \caption{Símbolo do Circuito no Cadence}
    \label{fig:TudoJunto}
\end{figure}

\newpage

\subsection{Simulação DC}

A simulação DC foi realizada com o objetivo de verificar a polarização do circuito onde foi usada uma corrente de $I_b = 80 \text{$\mu$A}$, estando os resultados obtidos ilustrados na Figuras \ref{fig:BiasOP} e \ref{fig:otaOP}.
Estes resultados encontram-se igualmente nas Tabelas \ref{tab:dc_bias} e \ref{tab:dc_main}. 

A análise do ponto de funcionamento mostra que os transístores operam na região de saturação, esta região de operação é adequada, não comprometendo o desempenho.

Apesar da tensão $V_{dsat}$ dos transistores $M_{10}$ e $M_{11}$ ser pouco maior do que o requisito, foi considerado aceitável dado que não compromete com o Output Swing
nem o coloca fora da zona de saturação.

Olhando para os valores das correntes, existe vemos que existem algumas discrepâncias no valor das mesmas no circuito principal isto
deve-se à não idealidade dos transístores, sendo estas influenciadas pelas dimensões dos dispositivos e pelo circuito de polarização. 

Embora este facto não represente um problema crítico visto que o circuito funciona conforme o esperado e as 
correntes apresentam valores razoáveis para o seu correto funcionamento, não atingido os valores máximos desta topologia.

Os valores das simulações DC podem ser vistos abaixo: 

\begin{figure}[H]
    \centering
    \begin{subfigure}[b]{0.49\textwidth}
        \centering
        \includegraphics[width=\textwidth]{Images/BiasOP.png}
        \caption{Simulação DC para circuito de Bias}
        \label{fig:BiasOP}
    \end{subfigure}
    \begin{subfigure}[b]{0.49\textwidth}
        \centering
        \includegraphics[width=\textwidth]{Images/OTAOP.png}
        \caption{Simulação DC para circuito principal}
        \label{fig:otaOP}
    \end{subfigure}
    
    \caption{Resultados da Simulação DC para ambos os circuitos}
    \label{fig:circuitos_completos}
\end{figure}





% --- Tabela VIII: Bias Circuit ---

\begin{table}[H]

    \centering

    \caption{Resultados da Simulação DC para circuito de bias}

    \label{tab:dc_bias}

    \vspace{0.2cm}

    \renewcommand{\arraystretch}{1} % Espaçamento para ficar arejado



    \begin{tabular}{ccc}

        \toprule[1pt]

        \textbf{Transistor} & \boldmath{$V_{DSat}$} & \boldmath{$I_{D}$} \\

        \midrule

       

        $M_{B1}$ & $-155.7 \, \text{mV}$ & $-8 \, \mu\text{A}$ \\

        \hline

        $M_{B2}$ & $-155.7 \, \text{mV}$ & $-8.897 \, \mu\text{A}$ \\

        \hline

        $M_{B3}$ & $88.84 \, \text{mV}$ & $8.898 \, \mu\text{A}$ \\

        \hline

        $M_{B4}$ & $-155.7 \, \text{mV}$ & $-8.343 \, \mu\text{A}$ \\

        \hline

        $M_{B5}$ & $297.5 \, \text{mV}$ & $8.343 \, \mu\text{A}$ \\

        \hline

        $M_{B6}$ & $-346.9 \, \text{mV}$ & $-9.243 \, \mu\text{A}$ \\

        \hline

        $M_{B7}$ & $89.17 \, \text{mV}$ & $9.243 \, \mu\text{A}$ \\

       

        \bottomrule[1pt]

    \end{tabular}

\end{table}
% --- Tabela IX: Main Circuit (CFC OTA) ---

\begin{table}[H]

    \centering

    \caption{Resultados da Simulação DC para circuito principal}

    \label{tab:dc_main}

    \vspace{0.2cm}

    \renewcommand{\arraystretch}{1.3}



    \begin{tabular}{ccc}

        \toprule[1pt]

        \textbf{Transistor} & \boldmath{$V_{DSat}$} & \boldmath{$I_{D}$} \\

        \midrule

       

        $M_{1a}$ & $58.3 \, \text{mV}$ & $42.52 \, \mu\text{A}$ \\

        \hline

        $M_{1b}$ & $-78.56 \, \text{mV}$ & $-36.57 \, \mu\text{A}$ \\

        \hline

        $M_{2a}$ & $58.24 \, \text{mV}$ & $42.39 \, \mu\text{A}$ \\

        \hline

        $M_{2b}$ & $-78.69 \, \text{mV}$ & $-36.66 \, \mu\text{A}$ \\

        \hline

        $M_{3}$ & $-93.14 \, \text{mV}$ & $-31.85 \, \mu\text{A}$ \\

        \hline

        $M_{4}$ & $-96.53 \, \text{mV}$ & $-31.85 \, \mu\text{A}$ \\

        \hline

        $M_{5}$ & $82.2 \, \text{mV}$ & $31.85 \, \mu\text{A}$ \\

        \hline

        $M_{6}$ & $80.16 \, \text{mV}$ & $31.85 \, \mu\text{A}$ \\

        \hline

        $M_{7}$ & $81.48 \, \text{mV}$ & $68.42 \, \mu\text{A}$ \\

        \hline

        $M_{8}$ & $81.49 \, \text{mV}$ & $68.51 \, \mu\text{A}$ \\

        \hline

        $M_{9a}$ & $87.77 \, \text{mV}$ & $84.92 \, \mu\text{A}$ \\

        \hline

        $M_{9b}$ & $-156.6 \, \text{mV}$ & $-73.23 \, \mu\text{A}$ \\

        \hline

        $M_{10}$ & $-156.6 \, \text{mV}$ & $-74.38 \, \mu\text{A}$ \\

        \hline

        $M_{11}$ & $-156.6 \, \text{mV}$ & $-74.24 \, \mu\text{A}$ \\

       

        \bottomrule[1pt]

    \end{tabular}

\end{table}

\newpage

\subsection{Simulação AC}

A simulação AC foi realizada com o objetivo de verificar o GBW do circuito, estando o respetivo diagrama de Bode representado na Figura \ref{fig:Sim_AC}.

Os resultados obtidos encontram-se apresentados na Tabela \ref{tab:ac_results}, 
onde é possível verificar que todos os objetivos e restrições de projeto foram cumpridos. 
Importa notar que, embora as frequências do segundo e terceiro polos sejam ligeiramente inferiores ao desejável, a estabilidade do sistema é garantida através de uma margem de fase adequada. 
Estes parâmetros poderiam ser alvo de uma otimização adicional para maximizar o ganho DC ou o GBW, porém 
optou-se por manter o dimensionamento atual de modo a respeitar estritamente os requisitos e objetivos estabelecidos na \ref{tab:specs}.


\begin{figure}[htp]
    \centering
    \includegraphics[width=12cm]{Images/AC_Sim.png}
    \caption{Diagrama de Bode do circuito}
    \label{fig:Sim_AC}
\end{figure}



\begin{table}[ht]
    \centering
    \caption{Resultados obtidos a partir do Maestro no Cadence}
    \label{tab:ac_results}
    \vspace{0.2cm}
    \centering
    % Alterado de {cc} para {lcc} (Left, Center, Center) para melhor alinhamento
    \begin{tabular}{lcc} 
        \toprule[1.2pt]
        \textbf{Parâmetro} & \textbf{Não Otimizados} & \textbf{Após Otimização} \\
        \midrule
        
        Ganho DC & $62.85 \, \text{dB}$ & $62.89 \, \text{dB}$ \\
        \hline
        GBW & $52.44 \, \text{MHz}$ & $58.34 \, \text{MHz}$ \\
        \hline
        Margem de Fase & $71^{\circ}$ & $65.42^{\circ}$ \\
        \hline
        Output Swing & $704.3 \, \text{mV}$ & $686.1 \, \text{mV}$ \\
        \hline
        Potência Consumida & $308.7 \, \mu\text{W}$ & $306.4 \, \mu\text{W}$ \\
        \hline
        Excess-Noise Factor & $2.277$ & $2.145$ \\
        \hline
        Figure of Merit & $849.4 \, \text{MHz} \cdot \text{pF/mW}$ & $952.11 \, \text{MHz} \cdot \text{pF/mW}$ \\
        \hline  
        Área Total & $1045.68 \, \mu\text{$m^2$}$ & $1580.68 \, \mu\text{$m^2$}$ \\
        \bottomrule[1.2pt]
    \end{tabular}
\end{table}


\subsection{Layout}

Foi realizado um layout simplificado do circuito, para verificar a área ocupada pelo mesmo.
Esta foi a última etapa realizada no Cadence e quando começamos, compreendemos imediatamente o porquê de se usar W's pequenos
e a utilidade dos fingers. Após nos termos apercebido disto, para efetuar o Layout fizemos umas pequenas alterações
não mudando a veracidade dos resultados. No circuito principal dividimos o valor de W por 10 e colocámos 10 fingers.
Na figura \ref{fig:LayoutCirc} é apresentado o circuito com esta pequena mudança e na Figura \ref{fig:Layout} está o resultado do layout. 

\begin{figure}[H]
    \centering
    \includegraphics[width=12cm]{Images/LayoutCirc.png}
    \caption{Circuito com algumas alterações nos Fingers e Multiplicadores para o Layout}
    \label{fig:LayoutCirc}
\end{figure}


\begin{figure}[H]
    \centering
    \includegraphics[width=12cm]{Images/Layout.png}
    \caption{Layout simplificado do Circuito Modificado}
    \label{fig:Layout}
\end{figure}


Isto perfaz uma área total de implementação física de aproximadamente $40593\ \mu m^2$. 
Comparando este valor com a Tabela \ref{tab:ac_results}, observa-se que a área de layout é significativamente 
superior. Esta diferença é esperada, pois o layout (obtido através de placement semi-automático) contempla não apenas 
a área ativa dos transístores, mas também os espaçamentos obrigatórios pelas regras de desenho (DRC), 
anéis de guarda, contactos e o espaço reservado para as interligações metálicas.

\newpage

\subsection{Comparação com o Folded-Cascode OTA}

Com os resultados obtidos para o CFC OTA, torna-se pertinente efetuar uma comparação da sua Figura de Mérito (FoM) com a do \textit{Folded-Cascode} OTA, cuja topologia se encontra representada na Figura \ref{fig:Cascode} e que foi analisado no trabalho laboratorial anterior.

Para além das evidentes diferenças entre as duas topologias, destaca-se a diferença da capacidade de carga ($C_{load}$) imposta, esta era de $8\,\text{pF}$. 
Esta carga capacitiva elevada obrigou, na altura, a um aumento substancial da corrente de polarização para garantir a estabilidade e a largura de banda exigidas. 

Olhando apenas em termos de corrente, apesar da corrente $I_b$ do CFC ser menor, o consumo de potência vai ser muito maior, devido à quantidade de transistores que o CFC tem.
Todas as vantagens do CFC referidas na Introdução levam a uma desvantagem, o folded cascode é mais eficiente energeticamente, levando a um maior FoM.


\newpage
